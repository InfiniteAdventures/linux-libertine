% LaTeX test file for the libertine font.
%
% Lizenz
%
% $Id$
%
% Michael Niedermair m.g.n@gmx.de
%

\chapter{Lizenz}

\section{GPL}

Unsere Schriften sind frei im Sinne der GPL und der OFL, d.\,h. dass
Veränderungen an der Schriftart erlaubt sind unter der Bedingung, dass diese
wieder der Öffentlichkeit unter gleicher Lizenz freigegeben werden. Querdenker
behaupten oft, dass bei der Verwendung einer GPL-Schrift eingebettet in
beispielsweise eine PDF auch diese freigestellt werden müsse. Das ist natürlich
totaler Blödsinn, denn unser Meinung nach findet bei der Einbettung der
Schriftart in die PDF keine nennenswerte Veränderung der Schrift statt. Und da
diese, wie sie ist, zu jedem Zeitpunkt auf unserer Seite heruntergeladen werden
kann, braucht die PDF auch nicht freigestellt zu werden. Da die GPL für
Quelltexte und Programme entwickelt wurde, es sich bei TrueType-Schriften aber
eher um eine Art Beschreibungssprache handelt, muss der GPL-Inhalt etwas freier
interpretiert werden. Die Entwicklung einer eigenen Lizenz ist uns
verständlicherweise zu aufwändig. Dieser Absatz versteht sich als Zusatzlizenz
zur GPL. Seit der Version 2.1.9 steht LinuxLibertine auch unter der OFL, welche
etwaige Nutzungsprobleme aus dem Weg räumen sollte. Weitere Informationen zur
GPL ( Wikipedia, Lizenztext ) und zur OFL (Wikipedia, Lizenztext).


\section{OFL}

This Font Software is Copyright (c) 2003-2007, Philipp H. Poll (\url{http://linuxlibertine.sf.net/}).
All Rights Reserved.

"`Linux Libertine"' is a Reserved Font Name for this Font Software.

This Font Software is licensed under the SIL Open Font License, Version 1.0.
No modification of the license is permitted, only verbatim copy is allowed.
This license is copied below, and is also available with a FAQ at:
\url{http://scripts.sil.org/OFL}
\bigskip

\hrule\bigskip
SIL OPEN FONT LICENSE Version 1.0 - 22 November 2005\bigskip
\hrule\bigskip

\minisec{PREAMBLE}

The goals of the Open Font License (OFL) are to stimulate worldwide
development of cooperative font projects, to support the font creation
efforts of academic and linguistic communities, and to provide an open
framework in which fonts may be shared and improved in partnership with
others.

The OFL allows the licensed fonts to be used, studied, modified and
redistributed freely as long as they are not sold by themselves. The
fonts, including any derivative works, can be bundled, embedded,
redistributed and sold with any software provided that the font
names of derivative works are changed. The fonts and derivatives,
however, cannot be released under any other type of license.

\minisec{DEFINITIONS}

"`Font Software"' refers to any and all of the following:
\begin{itemize}
\item font files
\item data files
\item source code
\item build scripts
\item documentation
\end{itemize}

"`Reserved Font Name"' refers to the Font Software name as seen by
users and any other names as specified after the copyright statement.

"`Standard Version"' refers to the collection of Font Software
components as distributed by the Copyright Holder.

"`Modified Version"'  refers to any derivative font software made by
adding to, deleting, or substituting -- in part or in whole --
any of the components of the Standard Version, by changing formats
or by porting the Font Software to a new environment.

"`Author"' refers to any designer, engineer, programmer, technical
writer or other person who contributed to the Font Software.

\minisec{PERMISSION \& CONDITIONS}

Permission is hereby granted, free of charge, to any person obtaining
a copy of the Font Software, to use, study, copy, merge, embed, modify,
redistribute, and sell modified and unmodified copies of the Font
Software, subject to the following conditions:

\begin{enumerate}
\item Neither the Font Software nor any of its individual components,
      in Standard or Modified Versions, may be sold by itself.
\item Standard or Modified Versions of the Font Software may be bundled,
      redistributed and sold with any software, provided that each copy
      contains the above copyright notice and this license. These can be
      included either as stand-alone text files, human-readable headers or
      in the appropriate machine-readable metadata fields within text or
      binary files as long as those fields can be easily viewed by the user.
\item No Modified Version of the Font Software may use the Reserved Font
      Name(s), in part or in whole, unless explicit written permission is
      granted by the Copyright Holder. This restriction applies to all
      references stored in the Font Software, such as the font menu name and
      other font description fields, which are used to differentiate the
      font from others.
\item The name(s) of the Copyright Holder or the Author(s) of the Font
      Software shall not be used to promote, endorse or advertise any
      Modified Version, except to acknowledge the contribution(s) of the
      Copyright Holder and the Author(s) or with their explicit written
      permission.
\item The Font Software, modified or unmodified, in part or in whole,
      must be distributed using this license, and may not be distributed
      under any other license.
\end{enumerate}

\minisec{TERMINATION}

This license becomes null and void if any of the above conditions are
not met.

\minisec{DISCLAIMER}

THE FONT SOFTWARE IS PROVIDED "`AS IS"', WITHOUT WARRANTY OF ANY KIND,
EXPRESS OR IMPLIED, INCLUDING BUT NOT LIMITED TO ANY WARRANTIES OF
MERCHANTABILITY, FITNESS FOR A PARTICULAR PURPOSE AND NONINFRINGEMENT
OF COPYRIGHT, PATENT, TRADEMARK, OR OTHER RIGHT. IN NO EVENT SHALL THE
COPYRIGHT HOLDER BE LIABLE FOR ANY CLAIM, DAMAGES OR OTHER LIABILITY,
INCLUDING ANY GENERAL, SPECIAL, INDIRECT, INCIDENTAL, OR CONSEQUENTIAL
DAMAGES, WHETHER IN AN ACTION OF CONTRACT, TORT OR OTHERWISE, ARISING
FROM, OUT OF THE USE OR INABILITY TO USE THE FONT SOFTWARE OR FROM
OTHER DEALINGS IN THE FONT SOFTWARE.


\endinput
