% LaTeX test file for the libertine font.
%
% Einf�hrung
%
% $Id$
%
% Michael Niedermair m.g.n@gmx.de
%

\section{Einführung}

"`Buchstaben und Schriftarten haben zwei Eigenschaften: Auf der einen Seite sind
sie grundlegende
Elemente für Kommunikation und Fundament unser Kultur, auf der anderen Seite
sind sie Kunst- und Kulturgüter.

Man braucht im allgemeinen Leben nur den ersten Aspekt dieser Eigenschaften
betrachten,
wenn es jedoch um Software geht, dann merkt man schnell, dass Markenrechte und
Patente schon auf den fundamentalen Schriftarten lasten. Wir wollen Ihnen eine
Alternative bieten und haben dafür das Libertine-Projekt freier Schriftarten
gegründet."'

{\raggedleft Philipp H. Poll\par}

Wir arbeiten an einer Schriftartenfamilie im TrueType- und Type1-Format. Sie ist
entworfen, Ihnen eine Alternative zu den bekannten Schriftarten wie T*mes New
Roman zu geben. Wir schreiben freie Software und veröffentlichen unsere
Schriftarten unter den Bedingungen der General Public License -- GPL. Sehen Sie
dazu unseren Absatz weiter unten: Lizenz.

Desweiteren besteht unser Ziel darin möglichst viele Sprachen und Sonderzeichen
zu unterstützen. Zur Zeit unterstützen unsere Schriftarten die westlichen
Zeichensätze Latein, Griechisch, Kyrillisch, Internationale Lautschrift und ihre
Sonderzeichen. Die LinuxLibertine enthält zur Zeit über 2300 Buchstaben.
Darunter sind Ligaturen wie fi, ff, fl usw. sowie Medivalzahlen, proportionale
und römische Zahlen. Es gibt daneben noch fleurale Symbole, Pfeile, Klötzchen,
hoch- und tiefgesetzte Zahlen.  Wenn Sie etwas vermissen oder Fehler finden,
dann senden Sie uns bitte eine Nachricht.

\endinput
