% LaTeX test file for the libertine font.
%
% Einf�hrung
%
% $Id$
%
% Michael Niedermair m.g.n@gmx.de
%
\chapter{Schriftbild}

\section{Übersicht}

\subsection{normal}
\printFont{t1fxl}

\subsection{old style}
\printFont{t1fxlj}

\subsection{old SS}
\printFont{t1fxlo}

\subsection{fitted}
\printFont{t1fxlf}


% \section{Text}
%
% \subsection{T1-Encoding (normal)}
% \printFontText{t1fxl}
% \newpage
%
% \subsection{T1-Encoding (old style)}
% \printFontText{t1fxlj}
% \newpage
%
% \subsection{T1-Encoding (fitted)}
% \printFontText{t1fxlf}

\newpage
\section{Aufzählungen}
\subsection{Aufzählungen mit Nummer}

Für die "`normale"' Aufzählung steht die Umgebung \emph{xlenumerate} zur Verfügung.
Als obtionaler Parameter kann dabei der Startpunkt im Font verwendet werden.
%Siehe hierzu Abschnitt~\fontref{fxlc}{m}{n}{U}.


% \begin{xlenumerate}
% \item Punkt 1
% \item Punkt 2
% \item Punkt 3
% \begin{xlenumerate}[22]
% \item Punkt 3.1
% \item Punkt 3.2
% \item Punkt 3.3
% \item Punkt 3.4
% \end{xlenumerate}
% \item Punkt 4
% \begin{xlenumerate}[124]
% \item Punkt 4.1
% \item Punkt 4.2
% \item Punkt 4.3
% \item Punkt 4.4
% \end{xlenumerate}
% \end{xlenumerate}
%
% \begin{lstlisting}
% \begin{xlenumerate}
% \item Punkt 1
% \item Punkt 2
% \item Punkt 3
% \begin{xlenumerate}[22]
% \item Punkt 3.1
% \item Punkt 3.2
% \item Punkt 3.3
% \item Punkt 3.4
% \end{xlenumerate}
% \item Punkt 4
% \begin{xlenumerate}[124]
% \item Punkt 4.1
% \item Punkt 4.2
% \item Punkt 4.3
% \item Punkt 4.4
% \end{xlenumerate}
% \end{xlenumerate}
% \end{lstlisting}
%
% \newpage
% \subsection{Aufzählungen mit Buchstaben}
%
% Es lassen sich aber auch Buchstaben verwenden, in dem man den Startpunkt
% auf 65 (entspricht~\fxlcsymbol{65}) oder 97 (entspricht~\fxlcsymbol{97}) setzt.
%
%
%
% \begin{xlenumerate}[65]
% \item Punkt 1
% \item Punkt 2
% \item Punkt 3
% \begin{xlenumerate}[97]
% \item Punkt 3.1
% \item Punkt 3.2
% \item Punkt 3.3
% \item Punkt 3.4
% \end{xlenumerate}
% \item Punkt 4
% \end{xlenumerate}
%
% \begin{lstlisting}
% \begin{xlenumerate}[65]
% \item Punkt 1
% \item Punkt 2
% \item Punkt 3
% \begin{xlenumerate}[97]
% \item Punkt 3.1
% \item Punkt 3.2
% \item Punkt 3.3
% \item Punkt 3.4
% \end{xlenumerate}
% \item Punkt 4
% \end{xlenumerate}
% \end{lstlisting}

% \section{Br�che}
%
% Br�che k�nnen �ber �ber \verb|\xlfrac{<z�hler>}{<nenner>}| bzw. mit der Sternvariante
% dargestellt werden.
%
% XX\xlfrac{0123456789}{0123456789}XX
%
% \xlfrac*{123}{456}
%
% \xlfrac*{123}{5}
%
% \xlfrac*{123}{11123345}
%
% $\xlfrac*{123}{456}$
%
% $\xlfrac*{123}{5}$


\section{Libertine-Logo}

Das Logo wird mit \verb|\xllogo| angezeigt: {\Huge\xllogo}



\endinput
