% LaTeX test file for the libertine font.
%
% FontDoku
%
% $Id$
%
% Michael Niedermair m.g.n@gmx.de
%
\chapter{\LaTeX-Installation}

\section{Linux-te\TeX{} (debian/ubuntu)}

\begin{enumerate}
\item Erstellen Sie im \textit{tmp}-Verzeichnis ein Unterverzeichnis mit dem Namen \emph{libertine}.
\begin{lstlisting}
mkdir -p /tmp/libertine
\end{lstlisting}
\item Entpacken Sie das Font-Paket in das \textit{tmp/libertine}-Verzeichnis.\footnote{Alle Aktionen müssen als User \textit{root} ausgeführt werden!}
\begin{lstlisting}
cd /tmp/libertine
unzip libertine_latex_2008_01_08.zip
\end{lstlisting}

\item Kopieren Sie den kompletten Inhalt in Ihren lokalen \textit{texmf}-Baum.\\
Das Verzeichnis des lokalen texmf-Baums können Sie über den Aufruf von \textit{kpsewhich} ermitteln.
\begin{lstlisting}
kpsewhich -expand-var='$TEXMFLOCAL'
   -> /usr/local/share/texmf
cp -Rv * /usr/local/share/texmf/
\end{lstlisting}
\item Aktualisieren Sie den \textit{texmf}-Baum.

\begin{lstlisting}
mktexlsr
\end{lstlisting}

\item Erzeugen Sie einen Eintrag für die map-Datei im \textit{updmap}-Verzeichnis und aktualisieren Sie die map-Dateien.

\begin{lstlisting}
echo "Map libertine.map" >/etc/texmf/updmap.d/99libertine.cfg
update-texmf
update-updmap
updmap-sys
\end{lstlisting}

\item Wenn Sie alle Schnitte des Fonts nutzen, kann es sein, dass die Speichervariablen von \TeX{} angepasst werden müssen. Bei der Entwicklung haben wir folgende Werte verwendet. Danach müssen die Formate mit \textit{fmtutil-sys} neu erstellt werden.

\begin{lstlisting}
echo "main_memory=5000000" >/etc/texmf/texmf.d/00libertine.cnf
echo "font_mem_size=2000000" >>/etc/texmf/texmf.d/00libertine.cnf
echo "pdf_mem_size=524288" >>/etc/texmf/texmf.d/00libertine.cnf
echo "save_size=10000" >>/etc/texmf/texmf.d/00libertine.cnf
update-texmf
fmtutil-sys --all
\end{lstlisting}

\end{enumerate}




\newpage
\section{Aufruf}

Für das Einbinden steht das Paket \textit{libertine.sty} zur Verfügung.

\begin{lstlisting}
\usepackage{libertine}
\end{lstlisting}

Es definiert für \textit{rmdefault} die Schrift \textit{fxl} (normale Ziffern).

\minisec{Optionen}

\begin{description}[\setlabelphantom{scaled}]
\item [osf] Es werden anstelle der normalen Ziffern Medivalziffern bzw. Minuskelziffern verwendet.
\begin{lstlisting}
\usepackage[osf]{libertine}
\end{lstlisting}

\item [scaled] Der Font wird entsprechend skaliert.
\begin{lstlisting}
\usepackage[scaled=0.95]{libertine}
\end{lstlisting}

\item [ss] Es wird \textsc{ss} anstelle von \textsc{ß} verwendet.
\begin{lstlisting}
\usepackage[ss]{libertine}
\end{lstlisting}

\end{description}

Ansonsten können Sie jeden Teilbereich über z.\,B.
\begin{lstlisting}
\usefont{T1}{fxl}{m}{n}\selectfont
\end{lstlisting}
auswählen. Siehe hierzu auch die Fonttabellen.


\section{Verwendung von Glpyhennamen}

Mit dem Befehl \verb|\useTextGlyph{<fontname>}{<glyphname>}| kann jedes Glpyh im Font
verwendet werden.


\verb|{\Huge\useTextGlyph{fxl}{uni211A}}| = {\Huge\useTextGlyph{fxl}{uni211A}} \par
\verb|{\Huge\useTextGlyph{fxl}{uni263A}}| = {\Huge\useTextGlyph{fxl}{uni263A}} \par
\verb|{\Huge\useTextGlyph{fxl}{Tux}}| = {\Huge\useTextGlyph{fxl}{Tux}} \par


\section{Source}

Die Routinen, um die \TeX-Metriken zu erzeugen und die Dokumentation (in \LaTeX) etc. findet man unter der fontforge CVS-Verwaltung (siehe hierzu \url{http://linuxlibertine.cvs.sourceforge.net/linuxlibertine/}).

\section{Liste aller Glyphen}

Eine Liste aller Glyphen findet sich unter\\
\url{http://linuxlibertine.sourceforge.net/latex/libertineglyphlist.pdf}.

\section{Fonttabellen}

Eine Übersicht der Fonttabllen findet sich unter\\
\url{http://linuxlibertine.sourceforge.net/latex/libertinetabellenuebersicht.pdf}.

\section{Änderungen}

\begin{description}[\setleftmargin{3cm}\breaklabel\setlabelstyle{\usefont{T1}{fxl}{b}{n}\selectfont}]
\item [08. Januar 2008]
\begin{itemize}
\item Verwendung der SFD-Dateien 2.7.x.
\item Anpassung der Versionsnummer an die Font-Versionsnummer.
\item Verzeichnisebene 'texmf' aus Paket entfernt.
\item LGI (expertimental).
\item Zahlen für Brüche (\verb|\xlfrax|) (expertimental).
\item Parameter \emph{ss} hinzugefügt (SS anstelle des versalen ß).
\item Aufteilung der Dokumentation in mehrere Bereiche.
\item Hinzufügen des LGR-Encodings für griechisch (expertimental).
\item Fehler mit Aufrufparameter 'osf' beseitigt.
\end{itemize}
\item [11. Juni 2007]
\begin{itemize}
\item Umstellung der Basis-mtx-Erstellung von \emph{fontsinst} auf \emph{ExTeX-Afm2Mtx}
\item Parameter \emph{debug} hinzugefügt.
\item Parameter \emph{scaled} hinzugefügt.
\item Parameter \emph{osf} hinzugefügt.
\item Alle Glpyhen lassen sich jetzt über den Glpyhnamen auswählen.
\item Hinzufügen des versalen ß.
\item Verzeichnisstruktur auf \texttt{texmf/fonts/afm/public/libertine} etc. angepasst.
\item Verzeichnis \texttt{texmf/doc/fonts/libertine} angelegt.
\end{itemize}
\item[1. Mai 2007]
\begin{itemize}
\item Erste Alpha-Version.
\end{itemize}
\end{description}

\section{Danke}

Ein besonderer Dank für die Unterstützung geht u.\,a. an folgende Personen:

\begin{multicols}{2}
Berry, Karl\\
Burnus, Tobias\\
Dirr, Ulrich\\
Hellström, Lars\\
Niepraschk, Rolf\\
Thiel, Rainer\\
\end{multicols}

\endinput
