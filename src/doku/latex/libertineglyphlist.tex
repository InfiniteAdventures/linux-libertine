% LaTeX test file for the libertine font.
%
% $Id$
%
% Michael Niedermair m.g.n@gmx.de
%
\listfiles
\documentclass{fontdokuold}

\usepackage{palatino}
\usepackage[debug]{libertine}

\begin{document}
\thispagestyle{empty}

\begin{minipage}{\linewidth}\fontsize{36pt}{40pt}\fontseries{m}\fontshape{n}\FontLibertine
   \textcolor{red}{\useTextGlyph{fxl}{uni2619}}\quad%
   \fontsize{36pt}{40pt}\fontseries{b}\fontshape{n}\FontLibertine%
    Linux Libertine Open\par
   \hfill\fontsize{36pt}{40pt}\fontseries{b}\fontshape{n}\FontLibertine%
   Fonts Project\quad%
   \fontsize{36pt}{40pt}\fontseries{m}\fontshape{n}\FontLibertine%
   \textcolor{red}{\useTextGlyph{fxl}{uni2767}}\par
   \centering%
\end{minipage}

\vfill
\begin{center}
   \fontsize{46pt}{46pt}\fontseries{b}\fontshape{n}\FontLibertine%
   Liste aller Glyphen
\end{center}

\vfill
\begin{center}\fontsize{20pt}{18pt}\FontLibertine
Font: Philipp H. Poll\par \LaTeX-Einbindung: Michael Niedermair
\end{center}

\vfill
\begin{center}
{\fontsize{6cm}{6cm}\fontseries{m}\fontshape{n}\FontLibertine%
\useTextGlyph{fxl}{uniE00A}}%
\hfill\fontsize{20pt}{18pt}\FontLibertine\today
\end{center}
\newpage

% ----------------------------------------------
\chapter{Glyphliste}

% \glyphTabEntry{fxl}{A}
\newcommand{\glyphTabEntry}[2]{%
\ifGylphExists{#1}{#2}{%
{\large\texttt{#2}\hfill\Huge\strut\useTextGlyph{#1}{#2}\par}}{}
}

\section{Libertine}
\setlength{\columnsep}{1cm}
\begin{multicols}{2}
{\catcode`\_=12
\input{xlglyphlist.tex}
}
\end{multicols}

\newpage
\section{Biolinum}
\setlength{\columnsep}{1cm}
\begin{multicols}{2}
{\catcode`\_=12
\input{xbglyphlist.tex}
}
\end{multicols}


% ----------------------------------------------

\end{document}

\endinput
