\listfiles
\documentclass[a4paper,10pt,parskip,ngerman]{scrartcl}
\usepackage[T1]{fontenc}
\usepackage[utf8]{inputenc}
%\usepackage[scaled=0.82]{luximono}
%\usepackage{libertine}
%\usepackage{lmodern}
\usepackage{babel}
%\usepackage{microtype}
\usepackage{geometry}
\geometry{paperheight=240mm,paperwidth=170mm,
    tmargin=5mm
    ,textwidth=125mm
           ,textheight=196mm,
           ,rmargin=22mm
           ,heightrounded
           ,includeheadfoot
           ,headheight=5mm
           ,headsep=8mm
           ,foot=18mm
           ,marginparsep=2mm
           ,marginparwidth=19mm}
\let\Lkeyword\texttt
\let\Ldim\texttt
\let\LColor\texttt
\def\Largr#1{\texttt{(#1)}}
\def\Lcs#1{\texttt{\textbackslash#1}}

\begin{document}

In der Regel wird man für Festlegung der Titelfolie einer Präsentation
kein Template als Vorlage benutzen, da hier außer den
Angaben zum Titel nur noch ein Hintergrund oder Logo der
Firma/wissenschaftlichen Einrichtung zu definieren wäre.
Zu beachten ist, dass der Parameter \Lkeyword{titlefont} nicht nur für
den Titel einer jeden Folie, sondern auch
für die \emph{Titelzeile} der Folie, die durch \Lcs{maketitle} erstellt
wird, berücksichtigt wird. Im folgenden Beispiel wird
die rechte obere Ecke der Titelseitenbox auf den Punkt
\Largr{0.95\Lcs{slidewidth}, 0.9\Lcs{slideheight}} gelegt. Die gesamte
Breite der Box beträgt  0.8\Ldim{slidewidth}, jeder Text der länger ist,
wird entsprechend umbrochen. Der Font für
die Titelzeile wird unterschiedlich zum Autoren- und Datumstext gewählt.
Zusätzlich wird der gesamte Hintergrund komplett
mit der Farbe \LColor{black!10} gefüllt.


{\fontfamily{fxl}\selectfont

In der Regel wird man für Festlegung der Titelfolie einer Präsentation
kein Template als Vorlage benutzen, da hier außer den
Angaben zum Titel nur noch ein Hintergrund oder Logo der
Firma/wissenschaftlichen Einrichtung zu definieren wäre.
Zu beachten ist, dass der Parameter \Lkeyword{titlefont} nicht nur für
den Titel einer jeden Folie, sondern auch
für die \emph{Titelzeile} der Folie, die durch \Lcs{maketitle} erstellt
wird, berücksichtigt wird. Im folgenden Beispiel wird
die rechte obere Ecke der Titelseitenbox auf den Punkt
\Largr{0.95\Lcs{slidewidth}, 0.9\Lcs{slideheight}} gelegt. Die gesamte
Breite der Box beträgt 0.8\Ldim{slidewidth}, jeder Text der länger ist,
wird entsprechend umbrochen. Der Font für
die Titelzeile wird unterschiedlich zum Autoren- und Datumstext gewählt.
Zusätzlich wird der gesamte Hintergrund komplett
mit der Farbe \LColor{black!10} gefüllt.
}


{\fontfamily{lmr}\selectfont

In der Regel wird man für Festlegung der Titelfolie einer Präsentation
kein Template als Vorlage benutzen, da hier außer den
Angaben zum Titel nur noch ein Hintergrund oder Logo der
Firma/wissenschaftlichen Einrichtung zu definieren wäre.
Zu beachten ist, dass der Parameter \Lkeyword{titlefont} nicht nur für
den Titel einer jeden Folie, sondern auch
für die \emph{Titelzeile} der Folie, die durch \Lcs{maketitle} erstellt
wird, berücksichtigt wird. Im folgenden Beispiel wird
die rechte obere Ecke der Titelseitenbox auf den Punkt
\Largr{0.95\Lcs{slidewidth}, 0.9\Lcs{slideheight}} gelegt. Die gesamte
Breite der Box beträgt 0.8\Ldim{slidewidth}, jeder Text der länger ist,
wird entsprechend umbrochen. Der Font für
die Titelzeile wird unterschiedlich zum Autoren- und Datumstext gewählt.
Zusätzlich wird der gesamte Hintergrund komplett
mit der Farbe \LColor{black!10} gefüllt.
}

\end{document}
